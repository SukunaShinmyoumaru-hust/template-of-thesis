\section{绪\hspace{1em}论}
本章我们首先介绍了当前存储系统面临的挑战和技术发展趋势,然后分析了对象存储技术的产生及发展现状,
介绍了国内外在存储策略和智能存储领域的相关研究工作,并对本文的主要研究内容及工作意义作了具体说明。
(除总结那一章以外,每章都加一段引言,注意引言和本章小结内容是不同的)

\subsection{课题背景、目的与意义}
绪论主要介绍本文的选题背景:说明本设计课题的来源、目的、意义、应解决的主要问题及应达到的技术要求;
简述本课题在国内外发展概况及存在的问题,本设计的指导思想等。(原则上,这一章内容主要来自于开题报告)

论文中不允许出现“我”、“我们”、“本文”之类的词,可以用“本课题”或者“本研究”,甚至可以直接用无主句。

\subsubsection{研究背景与趋势}
\begin{enumerate}
    \item 参考文献必须全部在论文正文中按顺序引用
    \item 不得在标题处进行引用,引用出现在标点之前。
    \item 可以在引用处右上角加标注进行引用,也可以直接在正文中用“文献[3]指出……”这样的话语进行引用
    \item 参考文献在后面的【参考文献】中排列顺序按照在论文中第一次引用的先后顺序排列
    \item 多篇参考文献群引不超过3篇,可使用如下风格:[1,2,3]、[1][2][3]
\end{enumerate}

\subsubsection{面临的问题和挑战}

\subsubsection{课题目的与意义}

阐述最终的输出成果能够达到什么目标,比如什么样的性能或者提供什么样的功能之类。

\subsection{国内外研究现状}
通过大量文献阅读,对所研究内容进行综述,较为详细的说明研究内容的国内外现状,
建议按照时间轴分阶段说明或者按照原理/机制分类别说明,不论哪种方式,
都要对每个阶段/每个类别的工作原理、机制、试图解决的问题、解决了的问题和存在的不足进行阐述和分析,
该部分篇幅在3-4页为宜

一次引用建议不要超过三篇文献,文献引用按照:递增顺序、全部引用的原则进行引用

\subsection{论文的主要内容与结构}
\subsubsection{论文的主要内容}

\subsubsection{论文结构}
本文的主要内容如下:

第一章...

第二章...

\dots

第六章...


% 插入图片,如图\ref{fig:fig1}所示:

% \cfig{fig1}{0.5}{插入图片示例}

% 公式,如公式\ref{eq:1}所示:
% \begin{equation}
%     \label{eq:1}
%     f_2 = f_v + f_a + f_{\omega}
% \end{equation}

% 列表,如\ref{chart:1}所示:
% \begin{table}[!ht]
%     \centering
%     \caption{歪比巴伯}
%     \label{chart:1}
%     \begin{tabular}{ccc}
%     \toprule
%         A & B & C  \\ \midrule
%         I & 1 & 2  \\ 
%         II & 3 & 4 \\ \bottomrule
%     \end{tabular}
% \end{table}

% 还可以添加算法\ref{al:1}:
% \begin{algorithm}[ht]
%     \caption{123}\label{al:1}
%     \begin{algorithmic}[1]
%         \Require a
% 			\Ensure b
%             \State v
% 			\While{q}
%             \State 1
% 			\If {d}
%             \State 3
% 			\State \Return dd
%             \Else
%             \State 2
%             \EndIf	
% 			\EndWhile
%     \end{algorithmic}
% \end{algorithm}
